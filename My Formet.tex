\documentclass[12pt,a4paper]{article}
\usepackage[latin1]{inputenc}
\usepackage{amsmath}
\usepackage{amsfonts}
\usepackage{amssymb}
\usepackage{graphicx}
\usepackage[acronym]{glossaries}
%\usepackage{setspace}
\author{Md Shahrear Zaman}
\title{My work}
\begin{document}
	 \begin{center}
	 	{\textbf{\huge A Simple Mathematical Model of the Inequity of Income and Wealth Distribution: A hypothetical country and society }}	
	     \vspace{1.5cm}
	 	 \\ Md Shahrear Zaman
	 	 \bigskip
	 	 \\M.S.s in Economics
	 	 \\University of Chittagong
	 	 \\M.S.c student in Economics
	 	 \\University of Kiel
	 	 \medskip
	 	 \\student.eco86@gmail.com
	 	 \bigskip
	 	 \\November 2017
	 	 
	 	 %\\Passport No. BH0954814 
	 	\end{center}
	\newpage
	\tableofcontents
	\newpage
	\section{Abstract}
In this paper we will observe the extent of the inequity of income and wealth distribution in a hypothetical country and society among different classes of people with the flow of time by constructing a simple mathematical model.		
	%\newacronym{Error Term}{ET}
	%\newacronym{Rsiduals}{RE}
		
		%\begin{align}
		%& ErrorTerm & \qquad ET \\
		%& Residuals  &\qquad RE \\
		%&...........................
		 %\end{align}	

	\section{Introduction}
Poverty is a one of the major problem on the path of civilization...... One of the dimension is inequity of income and wealth distribution. 
\medskip
\\A mathematical model is not strong enough to explain the full extent of the income and wealth distribution in a society.....which will help us not only to understand the core of this problem but also will help the policy makers to take the proper action and steps.                
                   
  \section{Structure of The Model}
	In this simple mathematical model some restrictions have imposed ......   
	
	Let us consider, a country with three classes: the upper, middle and the lower, under the consideration of their income and wealth.
     
    1. The number of people in upper class is less than those in the middle and in the lower class.
    
    2. The number of people in middle class is lower than those in the lower class.
    
    3. The total amount of the income and wealth in upper class is more than those in the middle and in the lower class.
    
    4. The total amount of the income and wealth in the middle class is more than that in the lower class. 
    
    5. The percentage of expenditure in the upper class is less than those in the middle and in the lower class.
    
    6. The percentage of expenditure in the middle class is less than that in the lower class.		
		
			
	\section{Model}
	The number of people :
   \\Upper  class = r
   \\Middle class = d 
   \\Lower  class = p 
   \\Social restriction: $ r < d < p $
   \medskip 
   \\On average the amount of income and wealth of each person :
   \\Upper  class = x
   \\Middle class = y
   \\Lower  class = z
   \\Social restriction: $ x > y > z $
   \medskip 
   \\The percentage of each person's expenditure  :
   \\Upper  class = k
   \\Middle class = m 
   \\Lower  class = n 
   \\Social restriction: $ 0 < k < m < n < 1 $
   \medskip 
   \\Total Years = $N$
   \medskip
   \\Total population $= r + d + p$
   \medskip
   \\After $N$ years the Income and Wealth of the Upper class will be
   \\$=r{\ast}x{\ast}N$ 
   \\After $N$ years the Income and Wealth of the Middle class will be
   \\$=d{\ast}y{\ast}N$
   \\After $N$ years the Income and Wealth of the Lower class will be
   \\$=p{\ast}z{\ast}N$
   \medskip
   \\After $N$ years total expenditure of $(r + d + p)$ people will be 
   \\$=(r{\ast}k{\ast}x + d{\ast}m{\ast}y + p{\ast}n{\ast}z){\ast}N$
   \\After $N$ years, the people of the upper class, middle class and the lower class will get back $\alpha$, $\beta$ and $\mu$ percent from the total income accordingly.
   \medskip
   \\After $N$ years,
   \\Savings of the Upper  class $=r{\ast}x{\ast}N -r{\ast}k{\ast}x{\ast}N$
   \\Savings of the Middle class $=d{\ast}y{\ast}N -d{\ast}m{\ast}y{\ast}N$
   \\Savings of the Lower class  $=p{\ast}z{\ast}N -p{\ast}n{\ast}z{\ast}N$
   \medskip
   \\Social restriction: $1\geqslant\alpha\geqslant\beta\geqslant\mu\geqslant0$
   \medskip
   \\For each person in upper class, after $N$ years the income and wealth will be
   $= x{\ast}N-k{\ast}x{\ast}N + \frac{\alpha}{r}{\ast}(r{\ast}k{\ast}x + d{\ast}m{\ast}y + p{\ast}n{\ast}z){\ast}N$
   \medskip 
   \\For each person in middle class, after $N$ years the income and wealth will be
   \medskip
   $= z{\ast}N-m{\ast}z{\ast}N + \frac{\beta}{d}{\ast}(r{\ast}k{\ast}x + d{\ast}m{\ast}y + p{\ast}n{\ast}z){\ast}N$
  \\For each person in lower class after $N$ years the income and wealth will be
  $= y{\ast}N-n{\ast}y{\ast}N + \frac{\mu}{p}{\ast}(r{\ast}k{\ast}x + d{\ast}m{\ast}y + p{\ast}n{\ast}z){\ast}N$ 
    \begin{figure}
    	\centering
    	\includegraphics[width=1\linewidth]{"Flow Chart2"}
    	\caption{Diagram}
    	\label{fig:flow-chart2}
    \end{figure} 
\newpage
  
  \section{Code}
  clear all
\\\% simulation: Example-1
\\ \%The number of people : Upper  class = NumU , Middle class = NumM, 
\\ \%Lower  class = NumL 
\\NumU = 2
\\NumM = 30
\\NumL = 500
\medskip 
\\ \%On average the amount of income and wealth of each person :   
\\ \%Upper class = IncomeWealthU , Middle class = IncomeWealthM,    
\\ \%Lower class =IncomeWealthL 
\\ \%IncomeWealthU $>$ IncomeWealthM $>$ IncomeWealthL 
\medskip 
\\ IncomeWealthU= 10000
\\ IncomeWealthM= 500
\\ IncomeWealthL= 100
\medskip
\\ TotalWealthIncome= IncomeWealthU*NumU + IncomeWealthM*NumM + IncomeWealthL*NumL
\medskip
\\ \%Number of the years
\\ N= 10
\\ \%After 10 years the income and wealth of the:
\\ IncomeWealthUNI=IncomeWealthU*NumU*N
\\ IncomeWealthMNI=IncomeWealthM*NumM*N
\\ IncomeWealthLNI=IncomeWealthL*NumL*N
\\ \%The percentage of each person's expenditure  : Upper  class = 
\\ \%PercentExpU  , Middle class = PercentExpM , Lower  class = PercentExpL  
\\ \%PercentExpU $>$ PercentExpM $>$ PercenrExpL
\medskip
\\ PercentExpU= 5/100 
\\ PercentExpM= 60/100 
\\ PercentExpL= 95/100
\medskip
\\ \%Total expenditure:
\\ TotalExp = NumU*IncomeWealthU*PercentExpU + NumM*IncomeWealthM*PercentExpM + NumL*IncomeWealthL*PercentExpL
\medskip
\\ \%After N years the total expenditure of the all of the people will be
\medskip
\\ N= 10
\\TotalExp= TotalExp*N 
\medskip
\\ \%After N years the fraction of the total expenditure that The people of 
\\ \%the upper class, the middle class and the lower class will get back:
\\ \%Upper  class = GetBU  , Middle class = GetBM, Lower  class = GetBL 
\\ \%GetBU$>$GetBM$>$GetBL
\\ \%GetBU+GetBM+GetBL=1
\medskip
\\GetBU= 60/100
\\GetBM= 30/100
\\GetBL= 10/100
\medskip
\\ if GetBU+GetBM+GetBL $>=$ 1
\\ \indent disp 'error: The sum of GetBU, GetBM and GetBL should be 1'
\\ end 
\medskip
\\ \%After N years the Income and Wealth of the every one in the upper, middle and the lower class
\medskip
\\NU= IncomeWealthU*N - PercentExpU*IncomeWealthU*N+ GetBU*TotalExpN/NumU  
\\NM= IncomeWealthM*N - PercentExpM*IncomeWealthM*N+ GetBM*TotalExpN/NumM  
\\NL= IncomeWealthL*N - PercentExpL*IncomeWealthL*N+ GetBL*TotalExpN/NumL 
\medskip
\\SavingsUpperClassN = (NumU*IncomeWealthU*N - NumU*IncomeWealthU*PercentExpU*N)
\\SavingMiddleClassN = (NumM*IncomeWealthM*N - NumM*IncomeWealthM*PercentExpM*N)
\\SavingLowerClassN =  (NumL*IncomeWealthL*N - NumL*IncomeWealthL*PercentExpL*N)
\medskip
\\TotalSavingsN = SavingsUpperClassN + SavingMiddleClassN + SavingLowerClassN
\medskip
\\TotalIncomeWealthN= IncomeWealthU*N*NumU + GetBU*TotalExpN - NumU*PercentExpU*IncomeWealthU*N +IncomeWealthM*N*NumM + GetBM*TotalExpN - NumM*PercentExpM*IncomeWealthM*N + IncomeWealthL*N*NumL + GetBL*TotalExpN - NumL*PercentExpL*IncomeWealthL*N 
\medskip
\\TotalSavingsAndExpenditureN=TotalSavingsN
+(PercentExpU*IncomeWealthU*N*NumU+
PercentExpM*IncomeWealthM*N*NumM
+PercentExpL*IncomeWealthL*N*NumL)
\medskip
\\subplot(3,0.5,1)
\\m =[IncomeWealthU,IncomeWealthM,IncomeWealthL];
\\bar(1:3,m,'m')
\\legend('Graph-1')
\\xlabel('1: Upper Class, 2: Middle Class, 3: Lower Class')
\\ylabel('Income and Wealth')
\\title('Comparison of the Income and Wealth Distribution at Present Period')
\\grid on
\\subplot(3,0.5,2)
\\k =[IncomeWealthUNI,IncomeWealthMNI,IncomeWealthLNI];
\\bar(1:3,k,'b')
\\legend('Graph-2')
\\xlabel('1: Upper Class, 2: Middle Class, 3: Lower Class')
\\ylabel('Income and Wealth')
\\title('Comparison of the Income and Wealth Distribution after 10 years without Inequality')
\\grid on
\\subplot(3,0.5,3)
\\y =[NU,NM,NL];
\\bar(1:3,y,'r')
\\legend('Graph-3')
\\xlabel('1: Upper Class, 2: Middle Class, 3: Lower Class')
\\ylabel('Income and Wealth')
\\title('Comparison of the Income and Wealth Distribution after 10 Years with Inequality')
\\grid on
 \section{Result}
 \begin{align}
 & Number\:of\: The\: People & \qquad NP \\
 & Percentage\: of\: Expenditure  &\qquad PE \\
 & Duration\: of\: Time  &\qquad DT \\
 & Percentage\: of\: Get\: Back  &\qquad PGB \\
 & Present\: Income\: and\: Wealth  &\qquad PIW \\
 & Future\: Income\: and\: Wealth\: Without\: Inequality\: &\qquad FIW \\
 & Future\: Income\: and\: Wealth\: With\: Inequality\:  &\qquad FIWI 
 \end{align}
\begin{tabular}{|c|c|c|c|c|c|c|c|}
	\hline 
	Class & NP & PE & DT & PGB & PIW& FIW & FIWI \\ 
	\hline 
	Upper & 2 & 5 & 10 & 60 & 1000 & 20000 &170300 \\ 
	\hline 
	Middle & 30 & 60 & 10 & 30 & 500 & 150000 &3160 \\ 
	\hline 
	Lower & 500 & 95 & 10 & 10 & 100 & 500000 &-736.80\\
	
\end{tabular}  
\newpage
\begin{figure}
\centering
\includegraphics[width=1\linewidth]{graph_a}
\caption{Comparision of The Income and Wealth Distribution}
\label{}
\end{figure}

\newpage
\section{Conclusion}
.......... restructured.
\newpage
\section{References}
    1. Zaman, Md Shahrear. 2011 .\textit{ Bohumatrik Daridrer Porimap: Manusher Jibon Jatonar Sarup Udghataner Ekti Prayogik Ebong Ganitik Prochesta} (Written in Bangla). Thesis Paper. Bangladesh: Department of Economics, University of Chittagong.
   \\2. User Manual. Octave.  
    
    
	 
\end{document}